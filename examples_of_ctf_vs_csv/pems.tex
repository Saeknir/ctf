The metadata in listing~\ref{lst:pems-metadata} describes the location, format, organization, and schema of the data.
Location in this case is many files on disk.
Format is compressed text, csv files.
Organization is by date that the sensor observations came in.
The schema and descriptions are in Listing~\ref{lst:pems-schema}.

\lstinputlisting[caption={\texttt{pems-metadata.json} contains the metadata for all the individual data files.}, label=lst:pems-metadata]{examples/pems/pems-metadata.json}

\lstinputlisting[caption={\texttt{pems-schema.json} contains the schemas for the tables.}, label=lst:pems-schema]{examples/pems/pems-schema.json}

The data may already physically organized so that we can do the \texttt{by()} computation.
If it was, then we would say this in listing~\ref{lst:pems-metadata}, perhaps by an entry such as \texttt{"partition": "station"}, where \texttt{"station"} is the name of the grouping column.
If it isn't grouped then we first need to group it, and this requires one complete pass through the data.
